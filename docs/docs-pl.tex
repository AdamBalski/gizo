\documentclass[11pt,a4paper]{article}

% Packages
\usepackage[utf8]{inputenc}
\usepackage[T1]{fontenc}
\usepackage[polish]{babel}
\usepackage{geometry}
\usepackage{graphicx}
\usepackage{hyperref}
\usepackage{listings}
\usepackage{xcolor}
\usepackage{booktabs}
\usepackage{amsmath}
\usepackage{float}
\usepackage{caption}
\usepackage{subcaption}

% Page geometry
\geometry{margin=1in}

% Code listing style
\definecolor{codebg}{RGB}{245,245,245}
\definecolor{codeframe}{RGB}{200,200,200}
\definecolor{codegreen}{RGB}{0,128,0}
\definecolor{codegray}{RGB}{128,128,128}
\definecolor{codepurple}{RGB}{128,0,128}

\lstdefinestyle{bashstyle}{
    backgroundcolor=\color{codebg},
    frame=single,
    rulecolor=\color{codeframe},
    basicstyle=\ttfamily\small,
    breaklines=true,
    captionpos=b,
    keepspaces=true,
    showspaces=false,
    showstringspaces=false,
    showtabs=false,
    tabsize=2,
    xleftmargin=0.5cm,
    framexleftmargin=0.5cm
}

\lstdefinestyle{pythonstyle}{
    backgroundcolor=\color{codebg},
    frame=single,
    rulecolor=\color{codeframe},
    basicstyle=\ttfamily\small,
    breaklines=true,
    captionpos=b,
    keepspaces=true,
    showspaces=false,
    showstringspaces=false,
    showtabs=false,
    tabsize=4,
    xleftmargin=0.5cm,
    framexleftmargin=0.5cm,
    keywordstyle=\color{codepurple},
    commentstyle=\color{codegreen},
    stringstyle=\color{codegray},
    language=Python
}

\lstdefinestyle{jsonstyle}{
    backgroundcolor=\color{codebg},
    frame=single,
    rulecolor=\color{codeframe},
    basicstyle=\ttfamily\footnotesize,
    breaklines=true,
    captionpos=b,
    keepspaces=true,
    showspaces=false,
    showstringspaces=false,
    showtabs=false,
    tabsize=2,
    xleftmargin=0.5cm,
    framexleftmargin=0.5cm
}

% Hyperref setup
\hypersetup{
    colorlinks=true,
    linkcolor=blue,
    filecolor=magenta,
    urlcolor=cyan,
    pdftitle={GIZO - Klasyfikacja białych krwinek},
    pdfauthor={Projekt GIZO},
}

% Title
\title{GIZO: Detekcja, klasyfikacja i adnotacja białych krwinek\\[0.5em]\large Dokumentacja użytkownika i raport techniczny}
\author{Projekt GIZO}
\date{\today}

\begin{document}

\maketitle

\begin{abstract}
GIZO to pipeline przetwarzania obrazów do wykrywania i klasyfikacji białych krwinek w obrazach mikroskopowych. System wykorzystuje segmentację opartą na kolorach do wykrywania regionów komórek oraz własną implementację klasyfikatora drzewa decyzyjnego ID3 wytrenowanego na 27-wymiarowych wektorach cech, aby klasyfikować komórki do czterech kategorii: limfocyty, eozynofile, neutrofile i monocyty. Niniejszy dokument zawiera dokumentację użytkownika dla systemu budowania i narzędzi adnotacji, a następnie raport techniczny opisujący podejście, algorytm i ewaluację wydajności.
\end{abstract}

\tableofcontents
\newpage

%===============================================================================
% CZĘŚĆ I: DOKUMENTACJA UŻYTKOWNIKA
%===============================================================================
\part{Dokumentacja użytkownika}

\section{Przegląd}

GIZO to całościowe narzędzie do analizy białych krwinek:
\begin{enumerate}
    \item \textbf{Pozyskiwanie danych} -- Pobieranie i przygotowanie zbioru danych Blood Cells z Kaggle
    \item \textbf{Detekcja} -- Lokalizacja białych krwinek przy użyciu segmentacji opartej na kolorach
    \item \textbf{Ekstrakcja cech} -- Obliczanie 27-wymiarowych wektorów cech dla każdej komórki
    \item \textbf{Klasyfikacja} -- Trenowanie i stosowanie klasyfikatora drzewa decyzyjnego ID3
    \item \textbf{Ewaluacja} -- Pomiar dokładności i generowanie wizualizacji wydajności
    \item \textbf{Adnotacja} -- Tworzenie obrazów z ramkami ograniczającymi i etykietami
\end{enumerate}

\section{Wymagania wstępne}

\subsection{Środowisko Python}
Projekt wymaga Pythona 3.10+ z następującymi zależnościami:
\begin{lstlisting}[style=bashstyle]
numpy>=2.0
pillow>=12.0
matplotlib>=3.10
\end{lstlisting}

Środowisko wirtualne jest automatycznie tworzone i zarządzane przez Makefile. Nie jest wymagana ręczna aktywacja - wszystkie cele Pythona automatycznie używają środowiska wirtualnego z katalogu \texttt{venv/}.

Jeśli chcesz ręcznie aktywować środowisko (np. do interaktywnej pracy):
\begin{lstlisting}[style=bashstyle]
source venv/bin/activate
\end{lstlisting}

\subsection{Zbiór danych}
System wykorzystuje zbiór danych Blood Cells z Kaggle. Do automatycznego pobrania potrzebne są skonfigurowane dane uwierzytelniające API Kaggle.

\section{Cele Make}

Projekt wykorzystuje GNU Make do automatyzacji budowania. Wszystkie cele są zdefiniowane w pliku \texttt{Makefile}. Poniżej znajduje się kompletna lista dostępnych celów:

\subsection{Przegląd wszystkich celów}

\begin{table}[H]
\centering
\begin{tabular}{ll}
\toprule
\textbf{Cel} & \textbf{Opis} \\
\midrule
\texttt{make venv} & Tworzy środowisko wirtualne i instaluje zależności \\
\texttt{make download} & Pobiera i rozpakowuje zbiór danych \\
\texttt{make data-trim} & Zmniejsza rozmiar zbioru danych (dla rozwoju) \\
\texttt{make create-bounding-boxes} & Wykrywa komórki w obrazach TRAIN \\
\texttt{make extract-features} & Ekstrahuje cechy z wykrytych komórek \\
\texttt{make eval} & Trenuje drzewo decyzyjne i ewaluuje na TEST \\
\texttt{make summary} & Generuje wykresy wydajności \\
\texttt{make render-decision-tree} & Wizualizuje strukturę drzewa \\
\texttt{make clean} & Usuwa wygenerowane artefakty \\
\texttt{make clean-data} & Usuwa pobrane dane \\
\texttt{make clean-venv} & Usuwa środowisko wirtualne \\
\texttt{make clean-all} & Czyści wszystko (dane, artefakty, venv) \\
\bottomrule
\end{tabular}
\caption{Kompletna lista celów Makefile}
\end{table}

\subsection{Cele zarządzania środowiskiem}

\subsubsection{\texttt{make venv}}
Tworzy środowisko wirtualne i instaluje wszystkie zależności z \texttt{requirements.txt}.

\begin{lstlisting}[style=bashstyle]
$ make venv
Creating virtual environment...
Installing dependencies...
Virtual environment ready!
Virtual environment is ready at venv
Activate with: source venv/bin/activate
\end{lstlisting}

\textbf{Wyjście:} Środowisko wirtualne w katalogu \texttt{venv/} z zainstalowanymi pakietami.

\subsubsection{\texttt{make clean-venv}}
Usuwa środowisko wirtualne.

\begin{lstlisting}[style=bashstyle]
$ make clean-venv
rm -rf venv
Removed virtual environment
\end{lstlisting}

\subsubsection{\texttt{make clean-all}}
Czyści wszystkie artefakty, dane i środowisko wirtualne.

\begin{lstlisting}[style=bashstyle]
$ make clean-all
rm -rf target
rm -rf data
rm -rf venv
\end{lstlisting}

\subsection{Cele zarządzania danymi}

\subsubsection{\texttt{make download}}
Pobiera i rozpakowuje zbiór danych Blood Cells z Kaggle.

\begin{lstlisting}[style=bashstyle]
$ make download
curl -L -o data/blood-cells.zip https://www.kaggle.com/...
unzip -q -o data/blood-cells.zip -d data
\end{lstlisting}

\textbf{Wyjście:} Rozpakowuje zbiór danych do \texttt{data}.

\subsubsection{\texttt{make data-trim}}
Zmniejsza rozmiar zbioru danych, zachowując tylko 10 najnowszych plików w każdym katalogu. Przydatne do szybszych iteracji podczas rozwoju.

\begin{lstlisting}[style=bashstyle]
$ make data-trim
\end{lstlisting}

\subsubsection{\texttt{make clean-data}}
Usuwa wszystkie pobrane dane.

\begin{lstlisting}[style=bashstyle]
$ make clean-data
rm -rf data
\end{lstlisting}

\subsubsection{\texttt{make clean}}
Usuwa wszystkie wygenerowane artefakty w katalogu \texttt{target/}.

\begin{lstlisting}[style=bashstyle]
$ make clean
rm -rf target
\end{lstlisting}

\subsection{Cele przetwarzania}

\subsubsection{\texttt{make create-bounding-boxes}}
Uruchamia \texttt{bounding\_boxes\_creation.py} w celu wykrycia białych krwinek w obrazach TRAIN i zapisania współrzędnych ich ramek ograniczających.

\begin{lstlisting}[style=bashstyle]
$ make create-bounding-boxes
python3 bounding_boxes_creation.py
Processing 9957 images with 12 worker(s)...
...
9957/9957
\end{lstlisting}

\textbf{Wyjście:} \texttt{target/bounding-boxes.json}

\textbf{Przykładowe wyjście (skrócone):}
\begin{lstlisting}[style=jsonstyle]
{
  "images_root": "data/dataset2-master/dataset2-master/images",
  "splits": ["TRAIN"],
  "interrupted": false,
  "missing_boxes": 0,
  "items": [
    {
      "image": "data/.../TRAIN/EOSINOPHIL/_6_1230.jpeg",
      "label": "EOSINOPHIL",
      "boxes": [[151, 88, 214, 146]]
    },
    ...
  ]
}
\end{lstlisting}

\subsubsection{\texttt{make extract-features}}
Uruchamia \texttt{feature\_extraction.py} w celu obliczenia 27-wymiarowych wektorów cech dla każdej wykrytej komórki.

\begin{lstlisting}[style=bashstyle]
$ make extract-features
python3 feature_extraction.py
Processing 9944 images with 12 worker(s)...
...
9944/9944
\end{lstlisting}

\textbf{Wyjście:} \texttt{target/features.json}

\textbf{Przykładowe wyjście (skrócone):}
\begin{lstlisting}[style=jsonstyle]
{
  "features": [
    "bbox_width_ratio", "bbox_height_ratio", "bbox_aspect_ratio",
    "bbox_area_ratio", "blue_fraction", "nucleus_fraction",
    "eosin_fraction", "pale_fraction", "monocyte_fraction",
    "cell_fraction", "mean_red", "mean_green", "mean_blue",
    "std_red", "std_green", "std_blue", "mean_blue_dom",
    "mean_saturation", "purple_fraction", "nucleus_components",
    "nucleus_holes", "pale_components", "pale_euler",
    "blue_holes", "compactness", "moment_ratio", "moment_spread"
  ],
  "items": [
    {
      "image": "data/.../TRAIN/EOSINOPHIL/_6_1230.jpeg",
      "label": "EOSINOPHIL",
      "bbox": [151, 88, 214, 146],
      "features": [0.2, 0.246, 1.085, 0.049, 0.634, ...]
    }
  ]
}
\end{lstlisting}

\subsubsection{\texttt{make eval}}
Uruchamia \texttt{evaluator.py} w celu wytrenowania drzewa decyzyjnego i ewaluacji na obrazach TEST.

\begin{lstlisting}[style=bashstyle]
$ make eval
python3 evaluator.py
Evaluating 2487 TEST images with 12 worker(s)...
...
2487/2487
\end{lstlisting}

\textbf{Wyjście:} \texttt{target/eval-results.json}

\textbf{Przykładowe wyjście (skrócone):}
\begin{lstlisting}[style=jsonstyle]
{
  "test_root": "data/dataset2-master/.../TEST",
  "total": 2487,
  "processed": 2487,
  "correct": 1958,
  "accuracy": 0.787,
  "interrupted": false,
  "items": [
    {
      "image": ".../TEST/LYMPHOCYTE/_2_3048.jpeg",
      "expected": "LYMPHOCYTE",
      "predictions": ["LYMPHOCYTE"],
      "matched": true
    },
    ...
  ]
}
\end{lstlisting}

\subsubsection{\texttt{make summary}}
Uruchamia \texttt{plotter.py} w celu wygenerowania wizualizacji ewaluacji.

\begin{lstlisting}[style=bashstyle]
$ make summary
python3 plotter.py
Wrote summary plot to target/eval-summary.png (overall accuracy 78.73%)
\end{lstlisting}

\textbf{Wyjście:} \texttt{target/eval-summary.png} -- 4-panelowy rysunek z wykresami dokładności, precyzji, macierzą pomyłek i statystykami podsumowującymi.

\subsubsection{\texttt{make render-decision-tree}}
Uruchamia \texttt{tree\_plotter.py} w celu wizualizacji struktury wytrenowanego drzewa decyzyjnego.

\begin{lstlisting}[style=bashstyle]
$ make render-decision-tree
python3 tree_plotter.py
Wrote decision tree diagram to target/decision-tree.png
\end{lstlisting}

\textbf{Wyjście:} \texttt{target/decision-tree.png}

\section{Adnotator drzewa decyzyjnego}

Skrypt \texttt{decision\_tree\_annotator.py} jest głównym narzędziem użytkownika do adnotowania nowych obrazów klasyfikacjami białych krwinek.

\subsection{Jak to działa}

Adnotator wykonuje następujące kroki:

\begin{enumerate}
    \item \textbf{Wczytanie danych treningowych:} Odczytuje \texttt{target/features.json} w celu uzyskania wektorów cech i etykiet z obrazów TRAIN.
    
    \item \textbf{Trenowanie klasyfikatora:} Buduje drzewo decyzyjne ID3 z parametrami:
    \begin{itemize}
        \item Maksymalna głębokość: 8
        \item Minimalna liczba próbek w liściu: 5
        \item Minimalny zysk informacyjny: 0.001
        \item Próg czystości: 0.9 (wczesne zatrzymanie)
    \end{itemize}
    
    \item \textbf{Przetwarzanie obrazów TEST:} Dla każdego obrazu w \texttt{data/.../TEST/}:
    \begin{itemize}
        \item Zastosowanie segmentacji opartej na kolorach do wykrycia regionów komórek
        \item Ekstrakcja wektora cech dla każdej wykrytej komórki
        \item Klasyfikacja przy użyciu wytrenowanego drzewa
        \item Rysowanie ramek ograniczających i etykiet na obrazie
    \end{itemize}
    
    \item \textbf{Zapis adnotowanych obrazów:} Wyjście odzwierciedlone w \texttt{target/} z zachowaniem struktury katalogów.
\end{enumerate}


\subsection{Użycie}

Skrypt \texttt{decision\_tree\_annotator.py} obsługuje dwa tryby pracy:

\subsubsection{Tryb wsadowy (domyślny)}
Przetwarza wszystkie obrazy w katalogu \texttt{data/.../TEST/}:

\begin{lstlisting}[style=bashstyle]
$ make venv # create a virtual environment if it does not exist
$ venv/bin/python decision_tree_annotator.py
\end{lstlisting}

Alternatywnie, jeśli środowisko jest już aktywne:
\begin{lstlisting}[style=bashstyle]
$ source venv/bin/activate
$ python3 decision_tree_annotator.py
\end{lstlisting}

\subsubsection{Tryb pojedynczego pliku}
Przetwarza jeden obraz wejściowy i zapisuje wynik do określonego pliku wyjściowego:

\begin{lstlisting}[style=bashstyle]
$ venv/bin/python decision_tree_annotator.py <plik_wejsciowy> <plik_wyjsciowy>
\end{lstlisting}

Przykład:
\begin{lstlisting}[style=bashstyle]
$ venv/bin/python decision_tree_annotator.py input.jpg output_annotated.jpg
\end{lstlisting}

\subsubsection{Pomoc}
Wyświetla informacje o użyciu:
\begin{lstlisting}[style=bashstyle]
$ venv/bin/python decision_tree_annotator.py --help
Usage:
  python decision_tree_annotator.py                    # Process all TEST images
  python decision_tree_annotator.py <input> <output>  # Process single image
  python decision_tree_annotator.py --help             # Show this help
\end{lstlisting}

\subsubsection{Wymagania}
Przed uruchomieniem skryptu upewnij się, że:
\begin{enumerate}
    \item Cechy są wyekstrahowane (\texttt{make extract-features})
\end{enumerate}

\subsection{Przykładowe wyjście}

\begin{lstlisting}[style=bashstyle]
Annotated target/.../TEST/EOSINOPHIL/_7_5399.jpeg :: EOSINOPHIL:1
Annotated target/.../TEST/LYMPHOCYTE/_2_3048.jpeg :: LYMPHOCYTE:1
Annotated target/.../TEST/MONOCYTE/_0_1381.jpeg :: MONOCYTE:1
Annotated target/.../TEST/NEUTROPHIL/_0_208.jpeg :: NEUTROPHIL:1
\end{lstlisting}

Każda linia pokazuje ścieżkę wyjściową i podsumowanie wykrytych komórek (np. \texttt{LYMPHOCYTE:1} oznacza wykrycie jednego limfocytu).

\subsection{Kolory etykiet}

Adnotator używa innych kolorów dla każdego typu komórki:
\begin{itemize}
    \item \textbf{LYMPHOCYTE} -- Limonkowy
    \item \textbf{EOSINOPHIL} -- Pomarańczowy
    \item \textbf{NEUTROPHIL} -- Cyjan
    \item \textbf{MONOCYTE} -- Magenta
\end{itemize}

\section{Przykładowe artefakty}

\subsection{Podsumowanie ewaluacji}

Rysunek~\ref{fig:eval-summary} przedstawia wizualizację wyników ewaluacji wygenerowaną przez \texttt{make summary}.

\begin{figure}[H]
    \centering
    \includegraphics[width=0.95\textwidth]{assets/eval-summary.png}
    \caption{Podsumowanie ewaluacji pokazujące dokładność per klasa, precyzję, macierz pomyłek i ogólne statystyki.}
    \label{fig:eval-summary}
\end{figure}

\subsection{Wizualizacja drzewa decyzyjnego}

Rysunek~\ref{fig:decision-tree} przedstawia strukturę wytrenowanego drzewa decyzyjnego.

\begin{figure}[H]
    \centering
    \includegraphics[width=0.95\textwidth]{assets/decision-tree.png}
    \caption{Wizualizacja wytrenowanego drzewa decyzyjnego ID3. Węzły wewnętrzne pokazują nazwy cech i progi; węzły liści pokazują przewidywane etykiety klas.}
    \label{fig:decision-tree}
\end{figure}

\subsection{Przykłady adnotowanych obrazów}

Rysunek~\ref{fig:annotated} przedstawia przykłady adnotowanych obrazów TEST dla każdego typu komórki.

\begin{figure}[H]
    \centering
    \begin{subfigure}[b]{0.45\textwidth}
        \centering
        \includegraphics[width=\textwidth]{assets/annotated-lymphocyte.jpeg}
        \caption{Limfocyt (poprawnie sklasyfikowany)}
    \end{subfigure}
    \hfill
    \begin{subfigure}[b]{0.45\textwidth}
        \centering
        \includegraphics[width=\textwidth]{assets/annotated-eosinophil.jpeg}
        \caption{Eozinofil (poprawnie sklasyfikowany)}
    \end{subfigure}
    
    \vspace{0.5cm}
    
    \begin{subfigure}[b]{0.45\textwidth}
        \centering
        \includegraphics[width=\textwidth]{assets/annotated-monocyte.jpeg}
        \caption{Monocyt (poprawnie sklasyfikowany)}
    \end{subfigure}
    \hfill
    \begin{subfigure}[b]{0.45\textwidth}
        \centering
        \includegraphics[width=\textwidth]{assets/annotated-neutrophil.jpeg}
        \caption{Neutrofil (poprawnie sklasyfikowany)}
    \end{subfigure}
    \caption{Przykładowe adnotowane obrazy pokazujące wykryte białe krwinki z ramkami ograniczającymi i etykietami klas.}
    \label{fig:annotated}
\end{figure}

\newpage

%===============================================================================
% CZĘŚĆ II: RAPORT TECHNICZNY
%===============================================================================
\part{Raport techniczny}

\section{Wprowadzenie}

Niniejszy projekt bada zautomatyzowane podejście do klasyfikacji białych krwinek wykorzystujące klasyczne techniki widzenia komputerowego i klasyfikację drzewem decyzyjnym.

\subsection{Sformułowanie problemu}

Mając obrazy mikroskopowe rozmazów krwi, system musi:
\begin{enumerate}
    \item Wykryć regiony zawierające białe krwinki
    \item Sklasyfikować każdą wykrytą komórkę do jednego z czterech typów:
    \begin{itemize}
        \item \textbf{Limfocyt} -- Małe komórki z dużym, ciemnym jądrem
        \item \textbf{Eozinofil} -- Komórki z dwupłatowym jądrem i różowo-pomarańczowymi ziarnistościami
        \item \textbf{Neutrofil} -- Komórki z wielopłatowym jądrem
        \item \textbf{Monocyt} -- Duże komórki z nerkowatym jądrem
    \end{itemize}
\end{enumerate}

\subsection{Zbiór danych}

Wykorzystujemy zbiór danych Blood Cells z Kaggle zawierający obrazy mikroskopowe zorganizowane według typu komórki:
\begin{itemize}
    \item \textbf{Podział TRAIN:} Używany do ekstrakcji cech i trenowania klasyfikatora
    \item \textbf{Podział TEST:} Używany do ewaluacji
\end{itemize}

Wykorzystaliśmy pełny zbiór danych zawierający około 9944 obrazów TRAIN i 2487 obrazów TEST.

\section{Podejście}

Nasz pipeline składa się z czterech głównych etapów: detekcji, ekstrakcji cech, klasyfikacji i adnotacji.

\subsection{Detekcja komórek}

Białe krwinki są wykrywane przy użyciu segmentacji opartej na kolorach w przestrzeni RGB:

\begin{equation}
\text{maska}(x,y) = \begin{cases}
1 & \text{jeśli } B > 0.42 \text{ i } B - (0.45R + 0.55G) > 0.08 \text{ i } B > G \\
0 & \text{w przeciwnym razie}
\end{cases}
\end{equation}

gdzie $R$, $G$, $B$ są znormalizowanymi kanałami kolorów w $[0,1]$.

Ta maska celuje w regiony z dominującym niebieskim charakterystyczne dla jąder komórkowych barwionych powszechnymi barwnikami hematologicznymi (np. Wright-Giemsa). Maska jest udoskonalana przy użyciu zamknięcia morfologicznego w celu wypełnienia małych przerw.

Analiza spójnych składowych następnie wyodrębnia poszczególne regiony komórek. Składowe są filtrowane według:
\begin{itemize}
    \item \textbf{Minimalne pole:} $\max(600, 0.003 \times \text{pole\_obrazu})$ pikseli
    \item \textbf{Minimalna zwartość:} 0.42 (stosunek pikseli do pola ramki ograniczającej)
\end{itemize}

\subsection{Ekstrakcja cech}

Dla każdego wykrytego regionu komórki obliczamy 27-wymiarowy wektor cech:

\subsubsection{Cechy geometryczne (4)}
\begin{itemize}
    \item \texttt{bbox\_width\_ratio}: Szerokość ramki / szerokość obrazu
    \item \texttt{bbox\_height\_ratio}: Wysokość ramki / wysokość obrazu
    \item \texttt{bbox\_aspect\_ratio}: Szerokość / wysokość ramki ograniczającej
    \item \texttt{bbox\_area\_ratio}: Pole ramki / pole obrazu
\end{itemize}

\subsubsection{Frakcje masek kolorów (7)}
Obliczanych jest kilka masek kolorów w celu identyfikacji różnych struktur komórkowych:

\begin{itemize}
    \item \texttt{blue\_fraction}: Proporcja pikseli z dominującym niebieskim (ogólna detekcja komórek)
    \item \texttt{nucleus\_fraction}: Proporcja ciemnoniebieskich pikseli ($B > 0.5$, $B-R > 0.15$, $B-G > 0.15$)
    \item \texttt{eosin\_fraction}: Proporcja różowo-pomarańczowych pikseli (ziarnistości eozynofili)
    \item \texttt{pale\_fraction}: Proporcja bladoróżowych pikseli (cytoplazma)
    \item \texttt{monocyte\_fraction}: Proporcja szarawych pikseli (cytoplazma monocytów)
    \item \texttt{cell\_fraction}: Unia wszystkich masek związanych z komórką
    \item \texttt{purple\_fraction}: Proporcja fioletowych pikseli
\end{itemize}

\subsubsection{Statystyki kolorów (8)}
\begin{itemize}
    \item \texttt{mean\_red}, \texttt{mean\_green}, \texttt{mean\_blue}: Średnie wartości RGB
    \item \texttt{std\_red}, \texttt{std\_green}, \texttt{std\_blue}: Odchylenia standardowe RGB
    \item \texttt{mean\_blue\_dom}: Średnia z $B - 0.5(R+G)$ (dominacja niebieskiego)
    \item \texttt{mean\_saturation}: Średnia z $\max(R,G,B) - \min(R,G,B)$
\end{itemize}

\subsubsection{Cechy morfologiczne (8)}
\begin{itemize}
    \item \texttt{nucleus\_components}: Liczba spójnych składowych w masce jądra
    \item \texttt{nucleus\_holes}: Liczba dziur w masce jądra
    \item \texttt{pale\_components}: Liczba składowych bladego regionu
    \item \texttt{pale\_euler}: Liczba Eulera maski bladej (składowe - dziury)
    \item \texttt{blue\_holes}: Liczba dziur w masce niebieskiej
    \item \texttt{compactness}: $\text{obwód}^2 / \text{pole}$ (miara okrągłości)
    \item \texttt{moment\_ratio}: Stosunek wartości własnych macierzy kowariancji (wydłużenie)
    \item \texttt{moment\_spread}: Suma wartości własnych (rozproszenie)
\end{itemize}

\subsection{Algorytm klasyfikacji}

Implementujemy własne drzewo decyzyjne ID3 (Iterative Dichotomiser 3) o następujących cechach:

\subsubsection{Konstrukcja drzewa}

Drzewo jest budowane rekurencyjnie przy użyciu zysku informacyjnego jako kryterium podziału:

\begin{equation}
\text{Zysk}(S, A) = H(S) - \sum_{v \in \text{Wartości}(A)} \frac{|S_v|}{|S|} H(S_v)
\end{equation}

gdzie $H(S)$ jest entropią zbioru $S$:

\begin{equation}
H(S) = -\sum_{c \in \text{Klasy}} p_c \log_2(p_c)
\end{equation}

\subsubsection{Kryteria zatrzymania}

Węzeł staje się liściem (przewidującym klasę większościową) gdy:
\begin{itemize}
    \item Wszystkie próbki mają tę samą etykietę (czysty węzeł)
    \item Osiągnięto maksymalną głębokość (8)
    \item Węzeł ma $\leq 5$ próbek
    \item Najlepszy podział ma zysk informacyjny $< 0.001$
    \item Czystość węzła $\geq 0.9$ (90\% tej samej klasy)
\end{itemize}

\subsubsection{Hiperparametry}

\begin{table}[H]
\centering
\begin{tabular}{lrl}
\toprule
\textbf{Parametr} & \textbf{Wartość} & \textbf{Cel} \\
\midrule
\texttt{max\_depth} & 8 & Zapobieganie przeuczeniu \\
\texttt{min\_leaf} & 5 & Minimalna liczba próbek w liściu \\
\texttt{min\_gain} & 0.001 & Minimalny zysk informacyjny \\
\texttt{purity\_threshold} & 0.9 & Próg wczesnego zatrzymania \\
\bottomrule
\end{tabular}
\caption{Hiperparametry drzewa decyzyjnego}
\end{table}

\section{Wyniki}

\subsection{Ogólna wydajność}

Na pełnym zbiorze testowym (2487 obrazów):

\begin{table}[H]
\centering
\begin{tabular}{lr}
\toprule
\textbf{Metryka} & \textbf{Wartość} \\
\midrule
Łącznie ocenionych obrazów & 2487 \\
Poprawne klasyfikacje & 1958 \\
Ogólna dokładność & 78,73\% \\
\bottomrule
\end{tabular}
\caption{Ogólne wyniki ewaluacji}
\end{table}

\subsection{Wydajność per klasa}

\begin{table}[H]
\centering
\begin{tabular}{lrrrr}
\toprule
\textbf{Klasa} & \textbf{Łącznie} & \textbf{Poprawne} & \textbf{Dokładność} & \textbf{Uwagi} \\
\midrule
LYMPHOCYTE & 620 & 617 & 99,5\% & Najlepsza klasa \\
EOSINOPHIL & 623 & 398 & 63,9\% & Często mylony z innymi \\
MONOCYTE & 620 & 457 & 73,7\% & Często klasyfikowany jako NEUTROPHIL \\
NEUTROPHIL & 624 & 486 & 77,9\% & Często klasyfikowany jako EOSINOPHIL \\
\bottomrule
\end{tabular}
\caption{Rozbicie dokładności per klasa}
\end{table}

\subsection{Analiza pomyłek}

Kluczowe obserwacje z macierzy pomyłek:
\begin{itemize}
    \item \textbf{Limfocyty} są dobrze rozróżnialne dzięki charakterystycznie małemu rozmiarowi i dużemu ciemnemu jądru
    \item \textbf{Eozynofile} są często błędnie klasyfikowane jako limfocyty lub neutrofile
    \item \textbf{Monocyty} są często mylone z neutrofilami (podobna morfologia)
    \item \textbf{Neutrofile} wykazują wysoką pomyłkowość z eozynofilami (oba mają płatkowate jądra)
\end{itemize}

\section{Dyskusja}

\subsection{Czego się nauczyliśmy}

\begin{enumerate}
    \item \textbf{Detekcja oparta na kolorach jest skuteczna:} Maska z dominującym niebieskim niezawodnie wykrywa barwione jądra z minimalną liczbą fałszywych pozytywów.
    
    \item \textbf{Inżynieria cech:} 27-wymiarowy wektor cech przechwytuje kluczowe właściwości morfologiczne i kolorymetryczne, ale niektóre cechy (jak eosin\_fraction) mają wysoką wariancję.
    
    \item \textbf{Metody klasyczne mają ograniczenia:} Bez bardziej zaawansowanych technik rozróżnianie podobnych typów komórek (neutrofil vs eozinofil, monocyt vs neutrofil) pozostaje wyzwaniem.
    
    \item \textbf{Drzewa decyzyjne są interpretowalne:} Wizualizacja drzewa pokazuje dokładnie, które cechy napędzają decyzje klasyfikacyjne, pomagając w debugowaniu i zrozumieniu.
    
    \item \textbf{Rozmiar zbioru danych wpływa na generalizację:} Pełny zbiór danych z tysiącami próbek treningowych pozwala modelowi lepiej uchwycić zmienność wewnątrzklasową.
\end{enumerate}

\subsection{Ograniczenia}

\begin{itemize}
    \item \textbf{Dokładność:} 78,73\% jest satysfakcjonująca jak na ten projekt, ale wciąż mogłaby być lepsza
    \item \textbf{Wrażliwość na niezbalansowanie klas:} Wydajność znacznie różni się między klasami
    \item \textbf{Wrażliwość na barwienie:} Progi kolorów zakładają spójne protokoły barwienia
    \item \textbf{Założenie pojedynczej komórki:} Detekcja może mieć problemy z nakładającymi się komórkami
\end{itemize}

\section{Podsumowanie}

Niniejszy projekt demonstruje kompletny pipeline do klasyfikacji białych krwinek przy użyciu klasycznych technik widzenia komputerowego. Obecna dokładność (78,73\%) jest obiecująca, a system zapewnia:

\begin{itemize}
    \item Działający pipeline od początku do końca od surowych obrazów do adnotowanych wyników
    \item Interpretowalną klasyfikację drzewem decyzyjnym
    \item Wszechstronne narzędzia wizualizacji i ewaluacji
    \item Podstawę do przyszłych ulepszeń
\end{itemize}

Klasa limfocytów osiąga 99,5\% dokładności, pokazując, że podejście działa bardzo dobrze dla wyróżniających się typów komórek. Poprawa wydajności na morfologicznie podobnych klasach (neutrofil/eozinofil/monocyt) wymagałaby albo bardziej wyrafinowanych cech, albo podejść głębokiego uczenia.

Baza kodu jest modułowa, dobrze udokumentowana i gotowa do rozszerzenia. Cały kod źródłowy, wytrenowane modele i artefakty ewaluacji są odtwarzalne za pomocą dostarczonych celów Makefile.

\appendix

\section{Referencja cech}

Pełna lista 27 cech ekstrahowanych na komórkę:

\begin{enumerate}
    \item \texttt{bbox\_width\_ratio} -- Szerokość ramki / szerokość obrazu
    \item \texttt{bbox\_height\_ratio} -- Wysokość ramki / wysokość obrazu
    \item \texttt{bbox\_aspect\_ratio} -- Szerokość / wysokość ramki
    \item \texttt{bbox\_area\_ratio} -- Pole ramki / pole obrazu
    \item \texttt{blue\_fraction} -- Frakcja pikseli z dominującym niebieskim
    \item \texttt{nucleus\_fraction} -- Frakcja ciemnoniebieskich pikseli (jądro)
    \item \texttt{eosin\_fraction} -- Frakcja różowo-pomarańczowych pikseli (ziarnistości eozyny)
    \item \texttt{pale\_fraction} -- Frakcja bladoróżowych pikseli (cytoplazma)
    \item \texttt{monocyte\_fraction} -- Frakcja szarawych pikseli
    \item \texttt{cell\_fraction} -- Połączona frakcja maski komórki
    \item \texttt{mean\_red} -- Średnia wartość kanału czerwonego
    \item \texttt{mean\_green} -- Średnia wartość kanału zielonego
    \item \texttt{mean\_blue} -- Średnia wartość kanału niebieskiego
    \item \texttt{std\_red} -- Odchylenie standardowe kanału czerwonego
    \item \texttt{std\_green} -- Odchylenie standardowe kanału zielonego
    \item \texttt{std\_blue} -- Odchylenie standardowe kanału niebieskiego
    \item \texttt{mean\_blue\_dom} -- Średnia dominacja niebieskiego
    \item \texttt{mean\_saturation} -- Średnia saturacja koloru
    \item \texttt{purple\_fraction} -- Frakcja fioletowych pikseli
    \item \texttt{nucleus\_components} -- Spójne składowe w jądrze
    \item \texttt{nucleus\_holes} -- Dziury w masce jądra
    \item \texttt{pale\_components} -- Spójne składowe w masce bladej
    \item \texttt{pale\_euler} -- Liczba Eulera maski bladej
    \item \texttt{blue\_holes} -- Dziury w masce niebieskiej
    \item \texttt{compactness} -- Zwartość kształtu (obwód$^2$/pole)
    \item \texttt{moment\_ratio} -- Stosunek wartości własnych (wydłużenie)
    \item \texttt{moment\_spread} -- Suma wartości własnych (rozproszenie)
\end{enumerate}
\end{document}
